\documentclass[../main.tex]{subfiles}
\graphicspath{{\subfix{images/}}}

\begin{document}

\subsection{Points forts}
\subsubsection{Graphismes}
Textures jolies et interfaces intuitives et soignees
\subsubsection{Performances}
Jeu fluide partout sauf dans le builder
\subsubsection{Editeur de niveau}
\subsubsection{Enregistreur de niveau aleatoires}
\subsubsection{Menu d'options}
Le menu permettant de modifier les options est relativement complet
\subsubsection{Pack de textures}
Il est tres facile de changer les textures et d'importer son propre pack

\subsection{Points faibles}
\subsubsection{Generation aleatoire}
\subsubsection{Interface non responsive}
\subsubsection{Taille des niveaux}
La taille des niveaux est limitee a un carre de 15*15 blocs

\subsubsection{Parametres permanents}
La classe Settings.java permet de stocker des valeurs et de les recuperer dans un fichier. Nous voulions l'utiliser pour conserver les parametres utilisateur entre chaques redemarrages mais nous n'avons pas eu le temps de l'integrer. Cette classe utilise java.util.Properties afin de recuperer les variables sauvegardees, modifier les valeurs et stocker les modifications. \\

Nous avons decide de laisser cette classe malgre le fait qu'elle n'est pas utilisee. Elle sert actuellement de "preuve de concept" et pourra eventuellement etre utilisee plus tard si nous decidons de continuer de travailler sur le jeu.

\end{document}
