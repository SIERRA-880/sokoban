\documentclass[../main.tex]{subfiles}
\graphicspath{{\subfix{images/}}}

\begin{document}

\subsection{Points forts}
\subsubsection{Graphismes}
La plupart des éléments graphiques ont été réalisées par un membre du groupe, ce qui nous a permi d'être totalement libre
par rapport à la direction artistique de notre jeu.
Ainsi, le style retro wave a été choisie pour les graphismes et la musique de notre jeu.
C'est un style peu utilisé pour des jeux de sokoban et qui plait à tous les membres.
Au niveau des interfaces graphiques, ils ont été rendus épurés et le plus intuitifs possible.	

\subsubsection{Performances}
Jeu fluide partout sauf dans le builder

\subsubsection{Editeur de niveau}
L'éditeur de niveau est un mode supplémentaire permettant d'imaginer et créer facilement ses propres niveaux de sokoban.
Ce mode permet la création d'une map de maximum 15 cases de largeur et hauteur. 
L'éditeur est composé d'un menu contenant tous les types de cellules que l'utilisateur pourra placer sur sa map et d'une grille sur laquelle
il pourra cliquer pour changer la cellule ciblée par la cellule choisie.
\subsubsection{Enregistreur de niveau aleatoires}
\subsubsection{Menu d'options}
Le menu permettant de modifier les options est relativement complet
\subsubsection{Pack de textures}
Il est tres facile de changer les textures et d'importer son propre pack

\subsection{Points faibles}
\subsubsection{Generation aleatoire}
\subsubsection{Interface non responsive}
\subsubsection{Taille des niveaux}
La taille des niveaux est limitee a un carre de 15*15 blocs

\subsubsection{Parametres permanents}
La classe Settings.java permet de stocker des valeurs et de les recuperer dans un fichier. Nous voulions l'utiliser pour conserver les parametres utilisateur entre chaques redemarrages mais nous n'avons pas eu le temps de l'integrer. Cette classe utilise java.util.Properties afin de recuperer les variables sauvegardees, modifier les valeurs et stocker les modifications. \\

Nous avons decide de laisser cette classe malgre le fait qu'elle n'est pas utilisee. Elle sert actuellement de "preuve de concept" et pourra eventuellement etre utilisee plus tard si nous decidons de continuer de travailler sur le jeu.

\end{document}
