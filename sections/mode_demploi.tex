\documentclass[../main.tex]{subfiles}
\graphicspath{{\subfix{images/}}}

\begin{document}

Le projet utilise Gradle comme système d'automatisation permettant de gérer facilement les dépendances et la compilation du code java et de la javadoc.
\subsection{Gradle}
Nous avons ajouté deux task a Gradle, trouvable a la fin du fichier \textit{build.gradle}
\begin{itemize}
    \item checkMap
    \item movReplay
\end{itemize}
Ces task sont utilisables de cette maniere : \\

\textbf{checkMap} \\
Cette task permet de verifier si des maps au format \path{.xsb} sont au format attendu. \\
Premier argument - \textbf{f} ou \textbf{d} permettant de dire au programme si on veut l'exécuter sur un \textbf{file} ou sur un \textbf{directory} \\
Second argument - \textbf{path} \\
Exemple - \textbf{./gradlew checkMap --args="fapp/build/resources/main/levels/map1.xsb"} \\

\textbf{movReplay} \\
Cette task permet de rejouer un fichier \path{.mov} sur un fichier \path{.xsb}. Cette task va automatiquement chercher dans le repertoire \path{build/resources/main/levels}
et \path{build/resources/main/appdata/movements}. Et va écrire un fichier de sortie dans \path{build/resources/main/levels/save} \\
Premier argument - \textbf{map} La map au format \path{.xsb} \\
Second argument - \textbf{mov} Le fichier de movements au format \path{.mov} \\
Exemple - \textbf{./gradlew movReplay --args="ma1 mov1"}

\subsection{Dépendances}
ffmpeg - nécessaire sur les systèmes UNIX afin d'afficher correctement la vidéo de fond de l'écran principal.
\subsection{Guide d'utilisation}
Notre sokoban comporte 3 modes. \\

La campagne principale disponible via le menu play. Ce mode permet aux joueurs de jouer
15 niveaux prédéfinis qui se déloquent les uns après les autres. Afin de compléter
un niveau, il faut pousser toutes les caisses sur les croix. Le déplacement du personnage se fait
avec les touches \path{z,q,s,d} ou avec les boutons présents sur l'interface graphique. Ces touches 
peuvent être redéfinies via le menu option. \\

Les modes Random et Builder accessibles via le menu Arcade. Le mode Random génére 
une map aléatoire dont on peut choisir les dimensions ainsi que le nombre de boites à pousser. \\
Le mode Builder permet au joueur de designer ses propres niveaux de jeux. \\
L'interface de Builder propose un niveau vide, de dimension 15*15, où chaque case peut être modifiée 
en cliquant dessus après avoir séléctionné un type de case dans la barre de gauche. \\

En plus de ces modes, n'importe quel niveau de Sokoban peut également être joué si le joueur possède le fichier d'extension \path{.xsb}.
Il suffit copier celui-ci dans le répertoire \path{sokoban/app/src/main/resources/levels/save}.
Il apparaitra ensuite (après un redémarage du jeu) dans la liste déroulante de la section load dans le menu Arcade avec les 
niveaux créés par le joueur et les niveau Random enregistrés.

\end{document}
