\documentclass[../main.tex]{subfiles}
\graphicspath{{\subfix{images/}}}

\begin{document}

\subsection{Open source}
Nous avons décidé de rendre le projet entièrement open-source et de l'héberger sur GitHub. 
En effet, nous tenons énormement à ce mouvement et développer un programme fermé n'est pas une possibilité pour nous. \\

L'open-source représente l'avenir de l'informatique et de l'enseignement, tout le monde n'a pas le luxe de se payer différentes resources.
L'open-source représente aussi la garantie d'un code plus sûr, plus performant et plus flexible.
Nous espérons que notre projet pourra servir de support pour d'autres étudiants et nous les encourageons a faire de même 
afin de contribuer au mouvement FOSS.

\subsection{Tools}
sokoban.Engine.Tools contient des classes statiques. Elles sont utilisées à différents endroits du code, permettant de le simplifier en les appelant. \\

On peut par exemple lire un fichier, écrire un fichier, formater un fichier .xsb. Choses qui prennent de la place et qui auraient pollué le code principal. En les regroupant dans un package, le code est beaucoup plus structuré.

\subsection{Easter eggs}
Pour ne pas voir le projet que comme un travail et prendre plus de plaisir en le faisant, nous avons décidé de rajouter des fonctionalités qui nous amusaient. \\
La première se déclanche quand on clique trois fois sur l'écran d'acceuil et la deuxieme s'active dans les paramètres en cochant un case. \\
On vous laisse le plaisir de les découvir !

\newpage

\end{document}
