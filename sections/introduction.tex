\documentclass[../main.tex]{subfiles}
\graphicspath{{\subfix{images/}}}

\begin{document}

\subsection{Objectif}
Ce projet est réalisé pour le cours de projet d'informatique donné en BAC 1 Sciences Informatiques a l'université de Mons.
Il s'agit d'un jeu de sokoban écrit en java sur la version 11.
Cette version n'est pas la plus récente (java 15 était disponible lors de la réalisation du projet début 2021) mais c'est la version LTS (long term support) la plus récente.
Voulant rendre le jeu open source sur GitHub, il nous semblait important d'utiliser une version LTS afin que d'autres étudiants puissent utiliser et/ou analyser notre code dans les années a venir.

\subsection{Procédure}
Afin de rendre le projet realisable, nous avons du nous organiser serieusement. \\
Nous avons commencé par réfléchir a la logique du jeu et faire un premier prototype en Python afin de souligner les points importants avant de se mettre a la programmation du moteur de jeu. \\
Ensuite differents outils ont été créés, facilitant le developpement du moteur qui était deja assez complexe et permettant de se concentrer sur des fonctionalités plus avancées. \\
Une base solide etant établie, le developpement de l'interface graphique a pris part au projet. Sans pour autant laisser de côté des améliorations du moteur et des tools. \\
Nous avons termine par le debugage et l'ecriture du rapport.

\newpage

\end{document}
