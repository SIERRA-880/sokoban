\documentclass[../main.tex]{subfiles}
\graphicspath{{\subfix{images/}}}

\begin{document}

\subsection{Objectif}
Ce projet est réalisé pour le cours de projet d'informatique donné en BAC 1 Sciences Informatiques a l'université de Mons.
Il s'agit d'un jeu de sokoban écrit en java sur la version 11.
Cette version n'est pas la plus récente (java 15 était disponible lors de la réalisation du projet début 2021) mais c'est la version LTS (long term support) la plus récente.
Voulant rendre le jeu open source sur GitHub, il nous semblait important d'utiliser une version LTS afin que d'autres étudiants puissent utiliser et/ou analyser notre code dans les années a venir.

\subsection{Procédure}
Afin de rendre le projet realisable, nous avons du nous organiser serieusement. \\
Nous avons commencé par réfléchir a la logique du jeu et faire un premier prototype en Python afin de souligner les points importants avant de se mettre a la programmation du moteur de jeu. \\
Ensuite differents outils ont été créés, facilitant le developpement du moteur qui était deja assez complexe et permettant de se concentrer sur des fonctionalités plus avancées. \\
Une base solide étant établie, le developpement de l'interface graphique a pris part au projet. Sans pour autant laisser de côté des améliorations du moteur et des tools. \\
Nous avons terminé par le débugage et l'écriture du rapport.

\subsection{Organisation}
Pour rester bien organisé, il était important d'établir qui fait quoi. Nous nous sommes réparti les tâches de cette manière : \\

Pignozi Agbenda s’est occupé en majorité de l’interface graphique en JavaFX. Il s’est aussi renseigné sur FXML et le CSS mais nous n’avons pas utilisé ces technologies dans notre projet. \\

Théo Godin a pensé et construit la logique du jeu et l’aspect programmation orientée objet. Il a aussi développé la plus grosse partie du générateur de niveau et a aidé P.Agbenda pour la partie interface. \\

Ugo Proietti a développé les tools et a pensé la logique de la génération. C’est également lui qui s’est occupé de la logistique, c'est-à-dire la configuration et l’enseignement de Gradle et de git au groupe. \\

Nous avons collaboré en utilisant git et hébergé notre repo sur Github. Une branch était créée par fonctionnalité. Une fois terminée, nous faisons un pull request sur la branch main. Nous avons également utilisé la fonctionnalité Issues permettant de rester à jour sur les bugs détectés. L’utilisation de Github nous a énormément aidé dans l’organisation du projet. \\
Voici le lien du repo (qui restera en privé avant d’avoir eu l’autorisation du professeur et des assistants de le passer en publique) : 
\href{https://github.com/SIERRA-880/sokoban}{https://github.com/SIERRA-880/sokoban} \\

Discord a été le canal de communication privilégié durant le développement. \\

\newpage

\end{document}
